\documentclass{article}
\usepackage{mathrsfs}
\usepackage{amsmath}
\usepackage{amsthm}
\usepackage{amssymb}
\usepackage{graphicx}
\usepackage{color}
%\include{macros}
%\usepackage{floatflt}
%\usepackage{graphics}
%\usepackage{epsfig}


\theoremstyle{definition}
\newtheorem{theorem}{Theorem}[section]
\newtheorem{lemma}[theorem]{Lemma}
\newtheorem{proposition}[theorem]{Proposition}
\newtheorem{corollary}[theorem]{Corollary}

\newtheorem{answer}{Ans}[section]

\theoremstyle{definition}
\newtheorem*{defition}{Definition}
\newtheorem*{example}{Example}

\theoremstyle{remark}
\newtheorem*{remark}{Remark}
\newtheorem*{note}{Note}
\newtheorem*{exercise}{Exercise}

\setlength{\oddsidemargin}{-0.25 in}
\setlength{\evensidemargin}{-0.25 in} \setlength{\topmargin}{-0.25
in} \setlength{\textwidth}{7 in} \setlength{\textheight}{8.5 in}
\setlength{\headsep}{0.25 in} \setlength{\parindent}{0 in}
\setlength{\parskip}{0.1 in}

\newcommand{\homework}[4]{
\pagestyle{myheadings} \thispagestyle{plain}
\newpage
\setcounter{page}{1} \setcounter{section}{#4} \noindent
\begin{center}
\framebox{ \vbox{\vspace{2mm} \hbox to 6.28in { {\bf
AU311,~Pattern~Recognition~Tutorial (Fall 2019) \hfill Homework: #1} }
\vspace{6mm} \hbox to 6.28in { {\Large \hfill #1 \hfill} }
\vspace{6mm} \hbox to 6.28in { {\it Lecturer: #2 \hfill} }
\vspace{2mm} \hbox to 6.28in { {\it Student: #3 \hfill} }
\vspace{2mm} } }
\end{center}
\markboth{#1}{#1} \vspace*{4mm} }


\begin{document}

\homework{2. Classification}{Xiaolin Huang \hspace{5mm} {\tt
xiaolinhuang@sjtu.edu.cn}}{XXX
\hspace{5mm} {\tt xxx@sjtu.edu.cn } }{9}

%%%%%%%%%%%%%%%%%%%%%%%%%%%%%%%%%%%%%%%%%%%%%%%%%%%%%%%%%%%%%%%%%%%%
% Section 2.  Problem
%%%%%%%%%%%%%%%%%%%%%%%%%%%%%%%%%%%%%%%%%%%%%%%%%%%%%%%%%%%%%%%%%%%%

\section*{Problem 1}\label{problem:1}
Solve the following linear l2-SVM by SGD method,
\begin{eqnarray}\label{nu-SVR}
\min_{w, \rho, \xi} & & \frac{1}{2} \|w\|_2^2  + \frac{1}{m} \sum_{i=1}^m \xi^2_i \nonumber\\
\mathrm{s.t.} & & y_i(w^T x_i + b) \geq 1 - \xi_i\\
& &  \xi_i \geq 0, \forall i = 1, 2, \ldots m. \nonumber
\end{eqnarray}

Try your code on dataset ``magic04'' (data provided, the last column stands for the label ).

i)  report the classification accuracy on the test data and plot the training accuracy v.s. the SGD iteration.

ii) numerically find the best ratio of samples when calculating the SGD. (For example, to achieve certain accuracy with the shortest time.)


\section*{Problem 2}\label{problem:2}
There have been many variants of SVM for different purpose. The following is called $\nu$-SVM which can controls the ratio of support vectors. The primal formulation of $\nu$-SVM is given as
\begin{eqnarray}\label{nu-SVR}
\min_{w, \rho, \xi} & & \frac{1}{2} \|w\|_2^2 -\nu \rho + \frac{1}{m} \sum_{i=1}^m \xi_i \nonumber\\
\mathrm{s.t.} & & y_i(w^T x_i + b) \geq \rho - \xi_i\\
& &  \rho \geq 0, \xi_i \geq 0, \forall i = 1, 2, \ldots m. \nonumber
\end{eqnarray}
Please derive its dual problem and discuss the meaning of $\nu$. 


%%%%%%%%%%%%%%%%%%%%%%%%%%%%%%%%%%%%%%%%%%%%%%%%%%%%%%%%%%%%%%%%%%%%
% Reference
%%%%%%%%%%%%%%%%%%%%%%%%%%%%%%%%%%%%%%%%%%%%%%%%%%%%%%%%%%%%%%%%%%%%

\end{document}
